\begin{lemma}
    If $X \sim \mathcal{N}(0,1)$ then $Y =-X$ also follows standard normal distribution.
\end{lemma}
\proof
\begin{align}
    P(Y \leq u) &= P(-X \leq u) \\
    &= P(X > -u) \\
    &= 1 - P(X \leq -u) \\
    &= 1 - (1 - P(X \leq u) \\
    &= P(X \leq u) 
\end{align}
As the distribution is symmetric, 
\begin{align}
 P(X\leq -u)=P(X\geq u)= 1-P(X\leq u)   
\end{align} 
\begin{lemma}
    If $n$ is an even number and $g(x)$ is an odd function, then,
    \begin{enumerate}
        \item 
    \begin{multline}\label{indep/3/equality}
        \pr{g(X_1)>\sum_{k=2}^ng(X_k)} \\=
        \pr{g(X_1)<\sum_{k=2}^ng(X_k)} \\
        = \frac{1}{2}
    \end{multline}
    \item 
    \begin{multline}
        \pr{g(X_1)>\prod_{k=2}^ng(X_k)}\\=
        \pr{g(X_1)<\prod_{k=2}^ng(X_k)} = \frac{1}{2}
    \end{multline}
    
\end{enumerate}
\end{lemma}
\proof 
\begin{enumerate}
\item 
\begin{multline}
    \pr{g(X_1)>\sum_{k=2}^ng(X_k)}\\=
    \pr{g(-X_1)<\sum_{k=2}^ng(-X_k)}\\=
    \pr{g(X_1)<\sum_{k=2}^ng(X_k)}
\end{multline}
%
As the cases
\begin{align}
    g(X_1)>\sum_{k=2}^ng(X_k)
\end{align}
and
\begin{align}
    {g(X_1)<\sum_{k=2}^ng(X_k)}
\end{align}
are complementary to each other, 
\begin{align}\label{indep/3/sum}
 \pr{g(X_1)>\sum_{k=2}^ng(X_k)}=\frac{1}{2}    
\end{align}
%
\item Similarly, 
\begin{multline}
    \pr{g(X_1)>\prod_{k=2}^ng(X_k)}\\=
    \pr{g(-X_1)<\prod_{k=2}^ng(-X_k)}\\=
    \pr{g(X_1)<\prod_{k=2}^ng(X_k)}
\end{multline}
As they follow the same distribution, the above expression is true. Thus we have
\begin{align}
    %\label{indep/3/equality}
    \pr{g(X_1)>\prod_{k=2}^ng(X_k)}=
    \pr{g(X_1)<\prod_{k=2}^ng(X_k)}
\end{align}
%
As the cases
\begin{align}
    g(X_1)>\prod_{k=2}^ng(X_k)
\end{align}
and
\begin{align}
    {g(X_1)<\prod_{k=2}^ng(X_k)}
\end{align}
are complementary to each other and from 
 \eqref{indep/3/equality} we have
\begin{align}\label{indep/3/prod}
 \pr{g(X_1)>\prod_{k=2}^ng(X_k)}=\frac{1}{2}    
\end{align}
\end{enumerate}
\begin{enumerate}[label = (\Alph*)]
    \item 
    From \eqref{indep/3/sum} , taking $g(x)=x$,
    \begin{align}
        \pr{X_1>X_2+...+X_{10}}=\frac{1}{2}
    \end{align}
\item
From \eqref{indep/3/prod} taking $g(x)=x$
        \begin{align}
         \pr{X_1>X_2X_3...X_{10}}=\frac{1}{2}   
        \end{align}
\item 
From \eqref{indep/3/sum} taking $g(x)=\sin{x}$
        \begin{align}
         \pr{\sin{X_1}>\sin{X_2}+...+\sin{X_{10}}}=\frac{1}{2}   
        \end{align}
\item \begin{multline}
        \pr{\sin{X_1}>\sin{(X_2+...+X_{10})}}\\=\pr{\sin{(-X_1)}<\sin{(-X_2-...-X_{10})}}\\
        =\pr{\sin{X_1}<\sin{(X_2+...+X_{10})}}
    \end{multline}
    As they follow the same distribution, the above expression is true.
    Thus we have
    \begin{multline} 
    \pr{\sin{X_1}>\sin{(X_2+...+X_{10}})}\\
        =\pr{\sin{X_1}<\sin{(X_2+...+X_{10}})}    
    \end{multline}
    Also, as $X_1$ is a continuous random variable
    \begin{align}
       \pr{\sin{X_1}=\sin{(X_2+...+X_{10})}}=0
    \end{align}
     As the cases
     \begin{align}
      {X_1>X_2+...+X_{10}}   
     \end{align}and 
     \begin{align}
         {X_1<X_2+...+X_{10}}
     \end{align}are complementary to each other 
        \begin{align}
         \pr{\sin{X_1}>\sin{(X_2+...+X_{10}})}=\frac{1}{2}   
        \end{align}
\end{enumerate}