As $X_{i}$ for $i = 1,2,\dots,n$ are independent random variables we can use this property to state
   \begin{align}
    Var\brak{\sum_{i=1}^{n}g(X_{i})}=\sum_{i=1}^{n}Var(g(X_{i}))\label{conv/3/basic}
    \end{align}
    \begin{definition}\label{conv/3/definition_of_Random_sample}
     Random Sample: The random variables $X_{1},X_{2},X_{3},\dots,X_{n}$ are said to be random sample if
   \begin{enumerate}
       \item the $X_{i}$'s are independent \label{conv/3/point 1}
       \item $F_{X_{\large{1}}}(x)=F_{X_{\large{2}}}(x)=\dots=F_{X_{\large{n}}}(x)=F_X(x)$ \label{conv/3/point 2}
       \item $EX_i=EX=\mu<\infty$ \label{conv/3/point 3}
       \item$ 0 < \mathrm{Var}(X_i)=\mathrm{Var}(X)=\sigma^2<\infty$ \label{conv/3/point 4}
   \end{enumerate}
    \end{definition}
 
   Let $n$ = 2 and hence $X_{1}$ and $X_{2}$ are sequence of independent random variables and 
   \begin{align}
       Var(X_{1}) &= 1\\
       Var(X_{2}) &= 2\\
       Var(X_{1}) &\neq Var(X_{2}) \label{conv/3/variance inquality} 
   \end{align}
   The equation \eqref{conv/3/variance inquality} doesn't follow point\eqref{conv/3/point 4} in  definition\eqref{conv/3/definition_of_Random_sample} and hence the random variables are not a random sample.
 
\begin{enumerate}
    \item Expectation of $\Bar{X_{n}}$ and $S_{n}$
    \begin{align}
    E(\Bar{X_{n}}) &= E\brak{\frac{1}{n}\sum_{i=1}^{n}X_{i}}\\
                   &= \frac{1}{n}\sum_{i=1}^{n}E\brak{X_{i}}\\
                   &= \frac{1}{n}\sum_{i=1}^{n}\mu\\
                   &= \mu \label{conv/3/expectation of Xn}\\
      E(S_{n}) &= E\brak{\frac{\sum_{i=1}^{n}\frac{1}{i}X_{i}}{\sum_{i=1}^{n}\frac{1}{i}}}      \\
               &= \frac{1}{\brak{\sum_{i=1}^{n}\frac{1}{i}}}\sum_{i=1}^{n}E\brak{\frac{1}{i}X_{i}}\\
               &= \frac{1}{\brak{\sum_{i=1}^{n}\frac{1}{i}}}\sum_{i=1}^{n}\frac{\mu}{i}\\
               &= \mu   \label{conv/3/expectation of Sn}
   \end{align}
   
   From \eqref{conv/3/expectation of Xn} and \eqref{conv/3/expectation of Sn} we get option\eqref{conv/3/option 1} is correct.
   
   \item 
    Variance of $\Bar{X_{n}}$ and $S_{n}$ using \eqref{conv/3/basic}
    \begin{align}
    Var(\Bar{X_{n}})&= Var\brak{\frac{1}{n}\sum_{i=1}^{n}X_{i}}\\
                   &= \frac{1}{n^{2}}\brak{\sum_{i=1}^{n}Var(X_{i})}\\
                   &= \frac{1}{n^{2}}\brak{\sum_{i=1}^{n}i}\\
                   &= \frac{1}{2} + \frac{1}{2n}\label{conv/3/variance of Xn}\\
    Var(S_{n}) &= Var\brak{\frac{\sum_{i=1}^{n}\frac{1}{i}X_{i}}{\sum_{i=1}^{n} \frac{1}{i}}}
    \end{align}
    \begin{align}
               &= \frac{1}{\brak{\sum_{i=1}^{n}\frac{1}{i}}^{2}}\sum_{i=1}^{n}\frac{1}{i^{2}}Var\brak{X_{i}}\\
               &= \frac{1}{\brak{\sum_{i=1}^{n}\frac{1}{i}}^{2}}\sum_{i=1}^{n}\frac{1}{i^{2}}i\\
               &= \frac{1}{\sum_{i=1}^{n}\frac{1}{i}}\label{conv/3/Variance of Sn}
   \end{align}
   As $n$ is sufficiently large 
   \begin{align}
       Var(\Bar{X_{n}}) &= \frac{1}{2}\\
       Var(S_{n})       &= 0\\
       Var(S_{n}) &< Var(\Bar{X_{n}})\label{conv/3/option 2 ans}
   \end{align}
   from \eqref{conv/3/option 2 ans} we get option\eqref{conv/3/option 2} as correct.
   
   \item
   \begin{definition}
   
     Point Estimator : Let $\theta$ be an unknown fixed(non-random) parameter be estimated. To estimate $\theta$ we define a point estimator $\hat{\Theta}$ that is a function of the random sample  $X_{1},X_{2},X_{3},\dots,X_{n}$ i.e.,
     \begin{align}
       \hat{\Theta}=h(X_1,X_2,\cdots,X_n)
   \end{align}
   
   \end{definition}
   
   \begin{definition}
     Consistent Estimator : Let $\hat{\Theta}_1$,$\hat{\Theta}_2$ $\dots$,$\hat{\Theta}_n$,$\dots$ be a sequence of point estimators of $\theta$. We say that $\hat{\Theta}_n$ is a consistent estimator of $\theta$ , if 
   \begin{align} \label{conv/3/consistent estimator definition}
       \lim_{n \rightarrow \infty} P\big(|\hat{\Theta}_n-\theta| \geq \epsilon \big)=0, \textrm{ for all }\epsilon>0.
   \end{align}
   \end{definition}
  
   From \eqref{conv/3/variance inquality} as given data is not a random sample we don't define point estimator and hence option\eqref{conv/3/option 3} is incorrect.
   
   \item
   \begin{definition}
     Statistic : A statistic is a function T = r($X_{1},X_{2},\dots,X_{n}$) of the random sample $X_{1},X_{2},\dots,X_{n}$.
   \end{definition}
   \begin{definition}
    Sufficient Statistics : A statistic t = T(X) is sufficient for $\theta$ if the conditional probability distribution of data X, given the statistic t = T(X), doesn't depend on the parameter $\theta$.
   \end{definition}
   Equation \eqref{conv/3/variance inquality} suggests that given data is not a random sample we don't define statistic and hence option\eqref{conv/3/option 4} is incorrect.
\end{enumerate}
Hence option\eqref{conv/3/option 1} and option\eqref{conv/3/option 2} are correct.