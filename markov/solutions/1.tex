\begin{figure}[!ht]
  \centering
  \begin{tikzpicture}
         
               % Setup the style for the states
          \tikzset{node style/.style={state, 
                                      minimum width=2cm,
                                      line width=1mm,
                                      fill=gray!20!white}}
          % Draw the states
          \node[node style] at (3, 0)      (bull)     {1};
          \node[node style] at (0, -4)      (bear)     {2};
          \node[node style] at (6, -4) (stagnant) {3};
          % Connect the states with arrows
          \draw[every loop,
                auto=right,
                line width=0.7mm,
                >=latex,
                draw=orange,
                fill=orange]
             (bull)     edge[bend left=20]            node {$\frac{1}{2}$} (stagnant)
              (bull)     edge[bend right=20] node {$\frac{1}{4}$} (bear)
              
              
              (bull) edge[loop above]             node  {$\frac{1}{4}$} (bull)
              (bear) edge[loop below]             node  {1} (bear)
              (stagnant) edge[loop below]             node  {1} (stagnant);
      \end{tikzpicture}
      \caption{}
      \label{fig:markov/1}
  \end{figure}
  The given problem can be represented using Table \ref{ec9:table:1} and the Markov chain in Fig. \ref{fig:markov/1}.
% Given, a fair coin is tossed is tossed two times.
% Let's define a Markov chain $\{X_{n},n=0,1,2,\dots\}$, where $X_{n}\in S=\{1,2,3\}$, such that
\begin{table}[ht!]
\centering
\begin{tabular}{|c|c|}
    \hline
    State & Description \\
    \hline
    1 & $\cbrak{T, T}$\\[1ex]
    \hline
    2 &  $Y = \cbrak{T, H}$\\[1ex]
    \hline
    3 & $N = \cbrak{\cbrak{H, H},\cbrak{H, T}}$\\[1ex]
    \hline
\end{tabular}
\caption{States and their notations}
\label{ec9:table:1}
\end{table}
The state transition matrix for the Markov chain can be expressed as
\begin{align}
  P=\begin{blockarray}{cccccc}
  &2 & 3 & 1  \\
  \begin{block}{c[ccccc]}
    2 & 1 & 0 & 0  \\
    3 & 0 & 1 & 0 \\
    1 & 0.25 & 0.5 & 0.25  \\
  \end{block}
  \end{blockarray}
  \label{ec9:eq:1}
  \end{align}
%  
Clearly, the state $1$ is transient, while $2,3$ are absorbing. Comparing   \eqref{ec9:eq:1} with the standard form of 
the state transition matrix 
\begin{align}
\label{ec9:eq:2}
P=\begin{blockarray}{ccc}
&A & N \\
\begin{block}{c[cc]}
  A & I & O  \\
  N & R & Q \\
\end{block}
\end{blockarray}
\end{align}
where,
\begin{table}[ht!]
\centering
\caption{Notations and their meanings}
\label{ec9:table:2}
\begin{tabular}{|c|c|}
    \hline
    Notation & Meaning \\
    \hline
    $A$ & All absorbing states\\[1ex]
    \hline
    $N$ & All non-absorbing states\\[1ex]
    \hline
    $I$ & Identity matrix\\[1ex]
    \hline
    $O$ & Zero matrix\\[1ex]
    \hline
    $R,Q$ & Other submatices\\[1ex]
    \hline
\end{tabular}
\end{table}
From   \eqref{ec9:eq:1} and  \eqref{ec9:eq:2},
\begin{align}
\label{ec9:eq:r,q}
    R=\myvec{
    0.25 & 0.5\\
    },
    Q=\myvec{
    0.25\\
    }
\end{align}
The limiting matrix for absorbing Markov chain is
\begin{align}
\bar P=\myvec{
    I & O\\
    FR & O\\
    }
    \label{eq:markov/1/limit}
\end{align}
where
\begin{align}
F=\brak{I-Q}^{-1} = \brak{1-0.25}^{-1} = \frac{4}{3}
\label{eq:markov/1/fund}
\end{align}
is called the fundamental matrix of $P$. 
Upon substituting from \eqref{ec9:eq:r,q} in \eqref{eq:markov/1/fund},
\begin{align}
  F=\brak{1-0.25}^{-1} = \frac{4}{3}
  \end{align}
  %
  and
  \begin{align}
    FR=\myvec{\frac{1}{3} & \frac{2}{3}}
    \end{align}
  which, upon substituting in \eqref{eq:markov/1/limit} yields
\begin{align}
\bar P=\begin{blockarray}{cccccc}
&2 & 3 & 1 \\
\begin{block}{c[ccccc]}
    2 & 1 & 0 & 0\\ 
    3 & 0 & 1 & 0\\ 
    1 & \frac{1}{3} & \frac{2}{3} & 0\\    
\end{block}
\end{blockarray}
\end{align}
% The desired probability is given by $\bar{p}_{12}$
% A element $\bar p_{ij}$ of $\bar P$ denotes the absorption probability in state $j$, starting from state $i$. 

% \\ Then, the absorption probability in state 2 $\brak{\text{i.e getting output Y}}$ starting from state 1 is $\bar p_{12}$.
\begin{align}
\therefore \bar p_{12}=\frac{1}{3}
\end{align}
